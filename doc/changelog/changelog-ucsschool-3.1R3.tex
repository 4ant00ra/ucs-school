\section{General}
\begin{itemize}
\item Domaincontroller master (single server environment) and domaincontroller slave (multi server
  environment) will now create automatically a new DHCP DNS policy which will set the IP address of the
  corresponding school domain controller as DNS server. This will be done on fresh installations and
  during/after the update by calling the join scripts. To avoid this new default behaviour, set the
  \ucsUCRV{ucsschool/import/generate/policy/dhcp/dns/set\_per\_ou} to \emph{false} before the installation of
  UCS@school or before updating UCS@school (\ucsBug{31930}).
\end{itemize}

% \section{Import scripts}
% \begin{itemize}
% \end{itemize}
 
% \section{iTALC}
% \begin{itemize}
% \end{itemize}

\section{Domain services}

\subsection{Samba4}
\begin{itemize}
\item In case a Windows client is joined to a UCS@school Single Master with Active Directory / Samba 4 domain functionality and the machine account for the new client was not pre-created by an Administrator via UMC, the new machine account is now created automatically under the Organizational Unit (OU) of the school. In case several schools are hosted, the first one found in OpenLDAP is chosen. The order of the OUs considered can be adjusted manually by the \ucsUCRV{ucsschool/local/oulist} (\ucsBug{31443}).
\item During join of an UCS@school Samba 4 Slave PDC the \ucsUCRV{nameserver2} and \ucsCommand{\ucsBCindex{nameserver3}} are now checked if they point to a nameserver that advertises SRV records for Kerberos that are invalid for the UCS@school Samba 4 Slave PDC. The \ucsUCRV{} is automatically unset in this case (\ucsBug{31651}).
\end{ucsConsoleInput}
\end{itemize}

\section{Univention Corporate Client integration}
\begin{itemize}
\item A new UCR policy will be set by new package \ucsName{ucs-school-ucc-integration} on UCS@school
  servers. The policy sets the \ucsUCRV{ucc/proxy/http} which will be evaluated by the UCC clients and sets
  the HTTP proxy on the UCC systems accordingly (\ucsBug{31966}).
\item By default, the squid negotiate (kerberos) authentication is activated
in samba4 environments (\ucsBug{31995}).
\end{itemize}

% \subsection{OpenLDAP}
% \begin{itemize}
% \end{itemize}

% \subsection{LDAP ACL changes}
% \begin{itemize}
% \end{itemize}
 
% \subsection{LDAP schema changes}
% \begin{itemize}
% \end{itemize}

% \subsection{Listener/Notifier domain replication}
% \begin{itemize}
% \end{itemize}

% \subsection{Domain joins of UCS systems}
% \begin{itemize}
% \end{itemize}

% \subsection{Services for Windows}
% \begin{itemize}
% \end{itemize}

%\subsection{Univention S4 Connector}
%\begin{itemize}
%
%\end{itemize}

\section{Print services}
\begin{itemize}
\item Support for restricting cups printer access has been added to the computerroom module (\ucsBug{31902}).
\end{itemize}

\section{Proxy services}
\begin{itemize}
\item The temporary file for the automatically created \ucsName{squidGuard.conf} is now created in the same directory to prevent problems during inter-device renames (\ucsBug{31397}).
\item The file permissions of the \ucsName{squidGuard} configuration are now fully specified and therefore not dependent on the umask (\ucsBug{31397}).
\item The package \ucsName{univention-squid-kerberos} has been added to the
UCS@school repository and is installed together with the UCC integration
package  (\ucsBug{31995}).
\end{itemize}

\section{RADIUS}
\begin{itemize}
\item The WLAN 802.1x integration (\ucsName{ucs-school-radius-802.1x}) now disallows access when an account is locked or disabled (\ucsBug{31587}).
\end{itemize}

%\section{Univention Directory Manager modules}
%\begin{itemize}
%
%\end{itemize}

% \section{Univention Management Console}
%
%\subsection{Univention Management Console server}
%\begin{itemize}
%
%\end{itemize}

%\subsection{Univention Management Console web interface}
%\begin{itemize}
%
%\end{itemize}

% \subsection{Univention Management Console modules}
% \begin{itemize}
% \end{itemize}

% \subsubsection{Configuration wizard}
% \begin{itemize}
% \end{itemize}

\subsubsection{Computerroom module}
\begin{itemize}
\item The \ucsName{computerroom} module has been addapted to support \ucsName{Univention Corporate Client} computer (\ucsBug{31903}).
\end{itemize}

% \subsubsection{Helpdesk module}
% \begin{itemize}
% \end{itemize}


\section{Other changes}
\begin{itemize}
\item The \ucsNAme{Add computer} wizard has been adapted to import \ucsName{Univention Corporate Client} computer \ucsBug{31904}.
\item Various \ucsName{UMC} modules have been adapted to correctly handle group names with differing case sensitivity (\ucsBug{31663}).
\item The user logonscripts are now created with executable flag.
 It is neccessary to add the executable flag to currently existing scripts (e.g. chmod +x \$scriptpath/*).
 Default scriptpaths are \ucsFile{/var/lib/samba/netlogon/user} and \ucsFile{/var/lib/samba/sysvol/\$REALM/scripts/user} (\ucsBug{31889}).
\end{itemize}
